\section{使用spmc channel编写的应用程序用到的接口和数据结构}
\subsection{dedup-DetMP和DMR等应用程序中用到的接口}
1.在我们使用spmc channel来编写dedup,DMR(DMR是我们课题组做的一个确定性的MapReduce)等应用程序时,均使用到了
spmc channel提供的一些接口。这些接口总结如表\ref{tab:DetMPInter}所示。

\bluet{注:}在spmc channel的实现中,
表中列出的spmcchan\_t的类型为int,spaceid\_t和tid\_t的类型均为uint8\_t。
这里我们记使用spmc channel实现的dedup为dedup-DetMP,用pthreads+queue实现的dedup为dedup-pthreads。
\begin{table}[!h]
\begin{footnotesize}
  \caption{dedup-DetMP等应用程序使用到的spmc channel提供的接口}
  \begin{tabularx}{1.0\textwidth}{|p{0.3\textwidth}|p{0.63\textwidth}|}
  \hline
  \textbf{接口名称} & \textbf{接口描述}\\
  \hline
  void spmc\_init(int nthreads) & 初始化SPMC环境,nthreads指明了内部创建的space的数目。\\
  \hline  
  void spmc\_destroy() &这个接口用于注销spmc环境。\\
  \hline
  spaceid\_t thread\_alloc(tid\_t gtid) & 分配一个相对编号为gtid的线程 \\
  \hline
  tid\_t thread\_start(spaceid\_t idx, void *(*fn)(void *), void *arg) & 
  派发一个space编号为idx的线程去做fn任务,arg是传递给fn函数的参数。\\
  \hline
  void thread\_join(spaceid\_t idx) & 等待space编号为idx的线程完成任务\\
  \hline
  int chan\_alloc(int nino)&分配nino个channel,返回分配的channel的编号\\
  \hline
  int chan\_setprod(spmcchan\_t cino, spaceid\_t dsid, bool cons) & 
  将相对编号为dsid的线程设置为编号为cino的channel的生产者,cons表示当这个线程写完之后是否成为它的消费者。\\
  \hline
  int chan\_setcons(spmcchan\_t cino, spaceid\_t dsid) & 
  将相对编号为dsid的线程设置为编号为cino的channel的消费者\\
  \hline
  chansize\_t chan\_send(spmcchan\_t cino, void *buf, size\_t nbyte)& 
  将起始地址为buf的数据发送到编号为cino的channel中,发送的消息大小为nbyte字节。\\
  \hline
  size\_t chan\_recv(spmcchan\_t cino, void *buf)& 
  从编号为cino的channel中接收消息,接收的消息放在buf开始的内存中。\\      
  \hline
 \end{tabularx}
\label{tab:DetMPInter}
\end{footnotesize}
\end{table}

2.对于DMR而言,除了使用了表\ref{tab:DetMPInter}列出的接口外,
它还额外使用了表\ref{tab:DMRInter}列出的接口,使用这些接口的原因见
表中的接口描述部分。
\begin{table}[!h]
\begin{footnotesize}
  \caption{DMR额外向应用程序提供的接口}
  \begin{tabularx}{1.2\textwidth}{|p{0.32\textwidth}|p{0.81\textwidth}|}
  \hline
  \textbf{接口名称} & \textbf{接口描述}\\
  \hline
  void spmc\_set\_copyargs(int flag, int sz) & 应用程序调用这个接口,
  以向DMR表明某个阶段的线程产生的key或者value指向私有堆,
  而后面阶段的线程需要key或者value指针指向的内容。
  由于DMR底层是基于DetMP来实现的,而DetMP中每个线程是私有的,为了让某个线程可以看见另外一个线程
  更新的堆变量,我们增加了这个接口。\\
  \hline
  void spmc\_set\_shareargs(void *addr, int sz) & 应用程序调用这个接口,
  以向DMR表明某个全局变量在MapReduce过程中会发生改变(即MapReduce中派发
  的线程会更新这个全局变量),而主线程在MapReduce调用结束后需要读取这个
  全局变量的值。DMR底层基于的DetMP使用了进程模拟线程的方式来达到空间隔离,
  因此各个进程更新的全局变量会变成私有的区域,而父进程看不见。
  为了让父进程可以看见子进程更新的值,我们提供了这个接口。
  \\
  \hline
 \end{tabularx}
\label{tab:DMRInter}
\end{footnotesize}
\end{table}

\section{基于pthreads+queue实现spmc channel提供的接口}
\paragraph{\hei{我们基于pthreads+queue来实现spmc channel的\bluet{思路}为}}:
\begin{itemize}
\item{定义一个全局的数据结构来保存queue和线程id等信息。}
\item{复用表\ref{tab:DetMPInter}和表\ref{tab:DMRInter}中提供的接口,但是给出基于pthread+queue的实现。}
\end{itemize}


下面先给出queue的数据结构定义和它提供的接口、保存queue和线程id信息的数据结构定义,
然后给出spmc channel给出的接口和queue给出的接口的映射关系。

\subsection{\bluet{queue和保存全局信息的数据结构定义以及pthreads+queue提供的接口}}
\paragraph{\bluet{queue的数据结构定义:}}
\label{sec:que}
queue的数据结构定义如代码\ref{lst:que}所示。队列的每个元素为指针类型,即void *类型。
\begin{lstlisting}[caption={data struct of queue},label={lst:que},float]
typedef struct queue {
  int head, tail;//head and tail of queue.
  void ** data;//`存储数据项,每个数据项为指针,即void*类型。`
  int size;//队列的大小
  pthread_mutex_t mutex;//`和empty,full一样,均用于控制线程的并发访问`
  pthread_cond_t empty, full;
}queue_t;
\end{lstlisting}

\paragraph{\bluet{queue提供的接口:}}
queue提供的接口如表\ref{tab:que}所示,主要操作为队列的初始化,入队,出队。
\begin{table}[!h]
\begin{footnotesize}
\caption{queue提供的接口}
  \begin{tabularx}{0.85\textwidth}{|p{0.32\textwidth}|p{0.46\textwidth}|}
  \hline
  \textbf{queue提供的接口名称} & \textbf{描述}\\
  \hline
  void queue\_init(struct queue * que, int size)& 初始化一个容量为size的队列que。\\
  \hline
  void *enqueue(struct queue* que, void *addr) & addr为指向数据项的指针, enqueue将addr指针存放到队列que中。\\
  \hline
  void *dequeue(struct queue * que)  &从队列que中取出一个元素,返回值是一个指向数据项的指针。\\
  \hline
 \end{tabularx}
\label{tab:que}
\end{footnotesize}
\end{table}

\paragraph{\bluet{保存线程和队列信息的数据结构定义}}
\label{sec:global}
代码\ref{lst:global}列出了保存线程信息和队列信息的数据结构的定义。
在具体实现时,要定义global\_info\_t类型的全局变量,\redt{记为global\_vars}。
global\_vars中的每个queue和DetMP中的channel是一一对应的,具体表现为编号为cino的channel
对应着global\_vars.que[cino]队列。
同时,DetMP的中每个线程的spaceid和pthreads情况下的pthreadid也是一一对应的,
具体表现为DetMP中编号为spaceid的线程对应着pthreads中global\_vars.tids[spaceid]线程。

\begin{lstlisting}[caption={`保存线程和队列信息的数据结构`},label={lst:global},float]
typedef struct global_info {
  pthread_t* tids;//`保存创建的所有线程编号` 
  int nth;//`记录当前分配的线程数目`
  struct queue *que[MAX_QUEUENUM];//`保存创建的所有队列信息,`
  //`MAX\_QUENUM是一个宏,它表示为分配的队列的最大的数目,初始值为1024`
  int nqueues; //`que数组中已经分配的队列的编号`
}global_info_t;
\end{lstlisting}

\paragraph{\bluet{spmc DetMP中定义的数据类型的转换:}}
spmc中定义了一些数据类型,对于这些数据类型的处理如下:

\begin{itemize}
\item{spmcchan\_t: 使用宏定义让其变成int类型。\#define spmcchan\_t int。}
\item{tid\_t: 使用宏定义让其变成pthread\_t类型。 \#define tid pthread\_t。}
\end{itemize}

\subsection{\bluet{基于pthreads+queue实现DetMP和DMR提供的接口:}}
我们可以保证原来使用DetMP进行编写的代码不变的情况下,再额外添加一个DetMP-pthreads.c.c文件,
DetMP-pthreads.c.c文件中对DetMP提供的接口给出了pthreads+queue的实现。
表\ref{tab:MapInter}列出了对于表\ref{tab:DetMPInter},\ref{tab:DMRInter}给出的接口怎么基于pthreads+queue来实现。

\begin{table}[!h]
\begin{footnotesize}
  \caption{基于pthreads+queue实现DetMP和DMR提供的接口}
  \begin{tabularx}{1.2\textwidth}{|p{0.3\textwidth}|p{0.83\textwidth}|}
  \hline
  \textbf{spmc channel提供的接口名称} & \textbf{用pthreads+queue实现时对应的转换}\\
  \hline
  void spmc\_init(int nthreads) & 接口不变,内部实现为初始化全局变量global\_vars,
  主要包括根据传入的参数nthreads给tids成员变量开辟大小为sizeof(pthread\_t)*nthreads的空间,
  将nth和nqueues置为零。global\_var变量见\ref{sec:global}的介绍。\\
  \hline   
  void spmc\_destroy() &接口不变,内部实现为注销在spmc\_init接口中初始化的全局变量。 \\
  \hline
  spaceid\_t thread\_alloc(tid\_t gtid) & 
  接口不变, 接口的语义仍然返回线程的相对id。实现时,返回global\_vars中的nth的当前值。
  使用pthread+queue实现获得的线程编号和使用spmc DetMP获得的线程编号是相等的。\\  
  \hline
  tid\_t thread\_start(spaceid\_t idx, void *(*fn)(void *), void *arg) & 转换成pthread\_create,具体可以使用宏定义。如\#define thread\_start(a,b,c) pthread\_create(\&(global\_vars.tids[a]),NULL,b,c)\\
  \hline
  thread\_join(spaceid\_t idx) & 转换成pthread\_join,具体的实现可以用宏定义\#define thread\_join(a, b) pthread\_join(global\_vars.tids[a], b)\\
  \hline
  int chan\_alloc(int nino)& 内部返回一个队列的编号,即其在global\_vars中的que数组中的编号。
  队列的编号和channel的编号是相等的。\\  
  \hline
   int chan\_setprod(spmcchan\_t cino, spaceid\_t dsid, bool cons) & 接口不变,内部实现为空.
  \\
  \hline
  int chan\_setcons(spmcchan\_t cino, spaceid\_t dsid) & 接口不变,内部实现为空即可
  \\
  \hline
  chansize\_t chan\_send(spmcchan\_t cino, void *buf, size\_t nbyte); & 
  接口不变,接口的实现中将buf的数据放入到编号为cino的队列(即global\_vars的que[cino]队列)中,
  具体操作可以调用enqueue来完成操作。\\
  \hline
  size\_t chan\_recv(spmcchan\_t cino, void *buf); & 
  内部实现为从全局编号为cino的队列(即global\_vars的que[cino]队列)中取数据到buf中,具体可以
  使用宏定义,如\#define chan\_recv(a, b) (b = dequeue(global\_vars.que[cino]))\\ 
  \hline 
  void spmc\_set\_copyargs(int flag, int sz) & 接口不变,内部实现为空即可。因为在pthreads共享内存的情况下,从堆中分配出来的变量是共享的,而不像DMR的情况下是私有的。\\
  \hline
  void spmc\_set\_shareargs(void *addr, int sz) & 接口不变,内部实现为空即可。
   因为在pthreads共享内存的情况下,全局变量是所有线程都可见的。\\  
  \hline
 \end{tabularx}
\label{tab:MapInter}
\end{footnotesize}
\end{table}

\subsection{基于pthreads+queue实现的参数设置}

1.目前初始化队列时,一个队列的容量设置为1024*1024*sizeof(void*)。
这个值为dedup-pthreads版本中设置的值。
在做dedup-DetMP和dedup-pthreads对比实验时,应该DetMP中channel的大小和队列的大小设置成相同的值。

\subsection{基于pthreads+queue实现detmp接口在实现上遇到的问题}
\bluet{问题1.chan\_send的实现需要拷贝发送者发送过来的数据。 }

虽然我们用pthreads+queue来实现detmp的接口时,线程之间是共享内存的。
但是在chan\_send的内部实现时仍然需要从堆中分配出空间,并将发送者发送过来的数据
拷贝到这个分配的空间中。
chan\_send的代码如清单\ref{lst:chan_send}所示。

\begin{lstlisting}[{label=lst:chan_send},caption={Implementation of chan\_send}]
void chan_send(spmcchan_t cino, void *addr, size_t sz){
  char *data = (char*)malloc(sz);//`从堆中开辟一个空间去`
  //`存放发送者发送过来的数据`
  memcpy(data, addr, sz);
  enqueue(&(global_vars.que[cino]), data);//`将发送者发送过来的数据入队`
}		
\end{lstlisting}

这样实现的\bluet{原因}有两个:

1)dedup-detmp实现的代码中,存在一个线程发送给另一个线程的数据是栈变量的情况。
为了表述方便,我们这里假设线程A发送一个栈变量给线程B。
在这种情况下,如果不保存发送过来的变量的内容,
而只是保存变量的地址,则有可能会
出现在B线程读取这个变量的时候,这个变量已经被A线程释放了,因为
栈变量在函数调用结束时会被释放掉。

2)在dedup程序中存在如下模式的代码,见代码\ref{lst:heapOpr}。
\begin{lstlisting}[label={lst:heapOpr},caption={Heap Operation Pattern},numbers=left,float]
DataProcess(...){
	...
	anchor = (char*)malloc(...);
	chan_send(args->w_ports[qid], anchor, p-anchor);
	...
	free(anchor);
}
FindAllAnchors(...){
	...
	chan_recv(args->r_ports[qid], anchor);
	...
}
\end{lstlisting}
在代码\ref{lst:heapOpr}中,DataProcess线程从堆中分配空间,
并将它赋值给anchor,
然后使用chan\_send将anchor的内容发送给FindAllAnchors线程,
并最后调用free函数将其分配的空间释放掉。
而FindAllAnchors线程则使用chan\_recv来接收anchor的内容。

从有这种访问模式的代码来看,如果在用pthreads+queue来实现detmp的接口时不
拷贝发送数据的内容而是仅传递指针,
则有可能会出现FindAllAnchors在访问数据的时候,
该空间已经被DataProcess线程调用free函数给释放掉了。






