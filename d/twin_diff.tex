\subsection{twin-and-diff}
DMR中有两个应用程序需要使用到Twin-and-diff机制

1.matrix\_multiply:该应用程序的map()函数,处理输入数据,并将结果直接存放于最终结果data.output中,data.output指向主线程master的heap区域中一块空间;
如果master与slaver采用共享堆,多个线程会共享该区域,mapreduce计算结束之后,data.output便会收集各个线程的局部结果。
  \begin{lstlisting}
    data->output[x_loc*data->matrix_len + i] = value;//data
  \end{lstlisting}
但DMR中,由于采用进程模拟线程,各线程间是地址隔离的,heap是私有的,因此主线程master的data.output无法收集到各个slaver的处理结果


Twin-diff


\subsubsection{详细设计方案}
1.scope的范围保证page对齐,未填满的部分,0填充;
未保护的区域从下一个页开始分配,这样做的目的是:避免多余的保护;
如果不需要保护的对象落入保护区,可能造成不必要的开销,甚至影响结果的正确性

2.多次迭代,scope是全局变量,master在进行下次的mapreduce时,并不会清空scope的
内容,保护的范围依然是第一次调用之前保护的范围
如果重新设置,错误,而且heap allocator可能会使用之前释放的heap空间

3.对于application中没有heap共享对象时,无需进行protect,无需进行twin-and-diff,
application不会触发malloc,spmc\_s.heap=NULL;

因此在twin\_dif.c中,需要判断scope==NULL的情况,即无需进行twin-and-diff

\subsubsection{TODO}
父进程中给出保护范围<start, end>,通常是整个master的heap
子进程在保护区域的时候,不仅仅需要保护父进程的保护区域,同时还需要保护其祖先传递给它的地址范围
这两个范围线性地址空间,可能不连续,那么孙子进程就需要保护多块地址空间

\subsubsection{bug记录}
snapshot\_init()备份的应该是产生pagefault的页,而不是从fva开始的页,否则比对的时候两者不一致,这将导致结果错误
