\subsection{环境参数}
\subsubsection{编译选项}
1.定义一个编译参数xxx,编译时开启的方式如下:
\begin{lstlisting}
  make xxx=1
\end{lstlisting}
多个参数可组合
\begin{lstlisting}
  make xxx=1 yyy=1
\end{lstlisting}

2.DMR中的编译参数如下:
\begin{itemize}
  \item combiner:定义该参数,则开启combiner;默认情况下关闭 
  \item array:定义该参数,则采用数组实现的buffer;默认情况下使用hash表实现的buffer
  \item nosort:定义该参数,在hash表实现的buffer中key数组无需排序,并采用普通查找法;
  默认情况下排序,采用折半查找
  \item ncores:可以定义不同核数的版本;默认为32核
\end{itemize}


3.不同应用程序的编译选项\\
针对五个应用程序,DMR相比phoenix具有较好性能的策略如表\ref{dmr_strategy}所示
\begin{table}[htbp]
\caption{不同应用程序最佳策略选择}
\label{dmr_strategy}
\begin{tabular}{|l|l|}
\hline
application & dmr strategy\\
\hline
histogram & no combiner array buffer\\
\hline
linear\_regression & combiner hash buffer\\
\hline
string\_match & no combiner hash buffer\\               
\hline
word\_count & no combiner array buffer\\
\hline
pca & combiner hash buffer \\
\hline
\end{tabular}
\end{table}


\subsubsection{DMR需要的dlinux运行参数}
当应用程序的数据集较大时,需要改变dlinux中的参数,才能正确运行
\begin{itemize}
  \item 调整改变EMEM的大小:smpc.h
\begin{lstlisting}
#define EMEM_NPAGE  (1<<22) // 8G->16G
\end{lstlisting}
  \item 调整channel和元数据的大小:/libspmc/inc/spmc\_chan.h
\begin{lstlisting}
#define CHAN_MAXSHIFT   17  // channel size: 512KB ->128K
#define SPMC_METASIZE   (1<<23) // channel metadata size: 4MB->8MB
\end{lstlisting}
  \item 调整heap的小小:spmc\_malloc.c
\begin{lstlisting}
#define SPMC_HEAP_SIZE      (((uint64_t)PAGE_SIZE << 22)) // 4MB->16MB
\end{lstlisting}
\end{itemize}
