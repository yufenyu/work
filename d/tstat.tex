\subsection{插桩收集DMR时间信息说明}
为了详细分析DMR执行过程中各阶段的时间开销,我们对DMR进行插桩
\begin{description}
  \item[代码位置:] git库dmr-1.0branch
  \item[commit:] 26b7c9a7193d9e4de8e0679ef291b37238dbf788
  \item[源文件:] src/dmr\_stat.c, src/inc/dmr\_stat.h, inc/dmr\_wrapfuns.h(配置文件)
\end{description}

\subsubsection{重要的全局变量和数据结构}
1.结构体
\begin{lstlisting}
typedef struct dmr_stat {
#ifdef DMR_TIMING
    int tid;
    double runtime;
    double tstart[CALLLAST];
    // time and counts of syscall/lib function calls
    double calltime[CALLLAST];
    uint64_t ncalls[CALLLAST];
#endif
} dmr_stat_t;
\end{lstlisting}
dmr\_stat\_t中tstart[id]记录每次运行id函数时的开始时间,calltime[id]记录id的总运行时间,ncalls[id]记录id被调用的次数

2.全局变量
\begin{lstlisting}
dmr_stat_t *timeinfo = NULL;
dmr_stat_t *dmr_s_tstat = NULL;
\end{lstlisting}
timeinfo用于存放所有线程的时间信息,在map,reduce中,默认情况下map线程数等于reduce线程数,另加master线程,因此初始化时,设置(MAX\_THREADS * 2 + 1)个,并使用MAP\_SHARE的mmap,保证多个线程都可以都可以修改该变量。

dmr\_s\_tstat:每个线程都拥有一个该全局变量,用于指向timeinfo[TID]的位置,每个线程在运行前,调用init\_
\begin{lstlisting}
  void init_timeinfo()
{
#ifdef DMR_TIMING
    int threads = MAX_THREADS * 2 + 1;
    timeinfo = (dmr_stat_t *)mmap(NULL,
                    sizeof(dmr_stat_t) * threads,
                    PROT_READ|PROT_WRITE,
                    MAP_SHARED|MAP_ANONYMOUS,
                    -1, 0);
    /* for master */
    init_dmr_tstat(0); 
    //printf("dmr_s_tstat=%p\n", dmr_s_tstat);
#endif
}
\end{lstlisting}
注意以下几点:
\begin{enumerate}
  \item spmc\_set\_copyargs()是在map\_reduce()之前被调用,为保证其中的DMR\_MALLOC()收集时间信息时,
  不出现指针为空的segment fault,需要提前调用init\_timeinfo设置timeinfo
  \item 为了收集master的完整时间信息,需要在init\_timeinfo()中调用init\_dmr\_tstat(0),
  因为在spmcenv()调用之前,无法通过init\_dmr\_tstat(TID)来设置
  \item 最后在结束env\_fini()中timeinfo=NULL,因为有些pca和Kmeans是多次map\_reduce()的计算
\end{enumerate}

\subsubsection{内存时间的收集}
1.DMR中涉及动态内存的操作:malloc, calloc, realloc, free,使用宏定义来替换,
方便收集不用类型的操作使用的malloc的时间开销
\begin{lstlisting}
#define DMR_MALLOC(a, b) mem_malloc(a, b)
#define DMR_FREE(a, b) mem_free(a, b)
#define DMR_CALLOC(a, b, c) mem_calloc(a, (size_t)b, (size_t)c)
#define DMR_REALLOC(a, b, c) mem_realloc(a, (void *)b, (size_t)c)
\end{lstlisting}

2.配置文件中
\begin{lstlisting}
  mmxx(TIME, KV, mem_kv, 9) //define MEM_KV
\end{lstlisting}
mmxx()用于定义一个MEM\_KV标识符(枚举类型),在应用程序中DMR\_MALLOC(MEM\_KV, sizeof(int) * len)来统计

3.插桩
\begin{lstlisting}
void *mem_malloc (ID_t type, size_t size)
{
    MEM_BEGIN(type);//memory time
    void *temp = malloc (size);
    assert(temp);
    MEM_END(type);//memory time

    return temp;
}
\end{lstlisting}
使用MEM\_BEGIN(),MEM\_END()进行插桩收集数据

\subsubsection{时间的收集}
收集时间的信息,最终会存入到dmr\_tstat.csv文件中
1.配置文件中
\begin{lstlisting}
fnxx(TIME, map, map, 1)
\end{lstlisting}

2.插桩
\begin{lstlisting}
        BEGIN_TIMING(map);
        /* do map()*/ 
        map_worker_do_task(env, tid);
        END_TIMING(map);
\end{lstlisting}



