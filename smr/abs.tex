\begin{abstract}
Although the multicore chips have been widely used, utilizing multicore sources is still  a challenging task due to the difficulties of parallel programming.
Traditional parallel programming techniques require the programmer to manually manage many details, such as synchronization, load balance.
%solve many details of design challenge.
%The programmers need to manually manage many details, such as synchronization, load balance, by using the traditional parallel programming techniques. 
%while designing and implementing efficient, simply parallel programming model is challenging because of many details of design challenges.
%Traditional parallel programming techniques like Pthreads and OpenMPI leave programmers solving many details of design challenge.
By automatically managing concurrency, MapReduce, a simple and elegant programming model, alleviates the burden of programmer. 
Phoenix, a MapReduce library for multicore and multiprocess, demonstrates that applications written with MapReduce framwork can get competitive scalability and performance in comparison to those written with Pthreads.
%Nevertheless, implementing such runtime systems for large-scale, shared-memory systems can be challenging.
%And it shows that MapReduce is a promising model for scalable performance on mulitcore and multiprocess.
%And applications written with MapReduce framwork have competitive scalability and performance comparied to those written with Pthreads.
However, evaluation results show that Phoenix scales worse on a 32-core system.

This paper focuses on improving Phoenix in terms of scalability and performance.
First, we analyze some critical factors that limit the Phoenix runtime, such as contending for a shared address space per process and existing of the barrier between Map and Reduce phase.
%we analyze the limitations of scalability and performance. 
Then, we propose a novel multithreaded model, \myth, which avoids the contention through providing the isolated address spaces between threads.
Based on \myth, we design a scalable mapreduce, \myds, which breaks the barrier and improves the performance by adopting a new producer-consumer model.
% for pipelining Map and Reduce phase.
%MapReduce library, \myds, which is based on a scalable thread library, \myth.
%Finally, we evaluate \myds on a 32 CPU cores processor and the result demonstrates that  applications, like \codet{histgram}, \codet{word\_count} and \codet{pca}, can achieve better scalability and performance than Phoenix.
%Finally, we evaluate \myds on a 32 CPU cores processor and the result shows performance improvements , applications like \codet{histgram}, \codet{word\_count} and \codet{pca}, has
Finally, the experiments show that \myds can achieve better scalability and performance than Phoenix for \codet{histgram}, \codet{word\_count} and \codet{pca}.
Specially, performance improvements range from 9.0X to 26.7X when the number of cores is 32.

\end{abstract}
