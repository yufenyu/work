\begin{abstract}
The multicore chips have been  widely used, while utilizing multicore sources is still challenging because of the difficulties of parallel programming.
Traditional parallel programming techniques like Pthreads and OpenMPI leave programmers solving many details of design challenge.
%The programmers need to manually manage many details, such as synchronization, load balance, by using the traditional parallel programming techniques. 
%while designing and implementing efficient, simply parallel programming model is challenging because of many details of design challenges.
%Traditional parallel programming techniques like Pthreads and OpenMPI leave programmers solving many details of design challenge.
MapReduce, a simple and elegant programming model, can simplify parallel programming by automatically managing concurrency without further burdening the programmer. 
Phoenix, a MapReduce library for multicore and multiprocess, demonstrates that applications written with MapReduce framwork have competitive scalability and performance compared to those written with Pthreads.
%Nevertheless, implementing such runtime systems for large-scale, shared-memory systems can be challenging.
%And it shows that MapReduce is a promising model for scalable performance on mulitcore and multiprocess.
%And applications written with MapReduce framwork have competitive scalability and performance comparied to those written with Pthreads.
However, evaluation results show that the optimized Phoenix\cite{yoo2009phoenix2} significantly underperformed on an 4-chip, 32-cores system running x86 Linux.

This paper focuses on improving Phoenix in the term of scalability and performance from a new perspective.
First, we will analyze the limitations of scalability and performance. 
Then we propose a novel thread thread library, \myds, with better scale for multicore. 
Based on \myth, we design a modified version of MapReduce model, a scalable mapreduce model (\myds) for multicore.
%MapReduce library, \myds, which is based on a scalable thread library, \myth.
Furthermore, we exploit a novel producer-consumer model to pipeline map and reduce phase.
We evaluated \myds on a 32 CPU cores processor.
The evaluation results show that application like \codet{histgram}, \codet{word\_count} and \codet{pca} with \myds achieve up to speedup over Phoenix in term of scalability and performance.

%is a safe and efficient alternative to Phoenix. 
%In addition, evaluation results show 
%application like \textit{pca} and \textit{word\_count} have better 
%scalabiliy and performans than Phoenix. 

\end{abstract}
