\begin{abstract}
Designing and implementing efficient, simply
parallel programming model is challenging.
Traditional parallel programming techniques
like Pthreads and OpenMPI leave programmers
solving many details of design chanllenges.
Dynamic runtimes can simplify parallel programming 
by automatically managing concurrency without
further burdening the programmer. 
Nevertheless, implementing such runtime systems for large-scale, shared-memory systems
can be challenging.
Phoenix, a MapReduce library for multicore, 
show that MapReduce is a promising model 
for scalable performance on mulitcore and multiprocess.
Applications written with MapReduce framwork
have competitive scalability and performance
comparied to those written with Pthreads.


%Our previous work, DMR\cite{zhang2015dmr}, a deterministic MapReduce for multicore,
%is a safe and efficient alternative to Phoenix. 
%In addition, evaluation results show 
%application like \textit{pca} and \textit{word\_count} have better 
%scalabiliy and performans than Phoenix. 

This work presents \myds, a modified version of phoenix framwork 
(...)
We evaluated \myds on a 32 CPU cores process.
The results show that \myds achieves up to (NUM1) speedup over Phoenix.
\end{abstract}
