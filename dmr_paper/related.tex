\section{Related Work}
%两个方面:(1)面向多核的mapreduce的研究:Phoenix, mutis, MRPhi (2)linux scalability的问题
\label{sec:rela}
%MapReduce以及已有的MapReduce库
MapReduce is a popular distributed framework for massive-scale parallel data analysis developed by Google.
There are many existing implementation of MapReduce which adopt on the basic architecture and  programming model of originally Google's MapReduce.
Hadoop\cite{White2010Hadoop}, an open source implementation of MapReduce, has been adopted enterprises including Yahoo! and Facebook. 
Map-Reduce-Merge\cite{Yang2007MapReducemerge} is proposed for supporting joins of heterogeneous datasets directly by adding a Merge phase.
Dryad\cite{isard2007dryad} generalizes MapReduce into an acyclic dataflow graph.

The Phoenix, a MapReduce libray implementation on multicore platform, is the most relevant work to \myds. 
Phoenix demonstrates tha applications that fit the MapReduce model can perform competitively with parallel code using Pthreads.
\myds differs from Phoenix mainly on two point:
Phoenix need barrier between Map and Reduce phase, while \myds brokes barrier to speed up computing;
And \myds exploits \myth thread to implement map and reduce workers, which results in fewer contention in Linux kernel caused by a shared address space.

%多核环境下的mapreduce库,虽然我们没有与mites, Tilt-MapReduce进行对比,但通过我们的研究,只要这些库使用pthread进行编程,那么就会存在scalability较差的问题。
Tilt-MapReduce\cite{chen2010tiled} uses the “tiling strategy” to partition a large MapReduce job into a number of small sub-jobs and handles the sub-jobs iteratively.
MRPhi\cite{lu2013mrphi} is a MapReduce framework optimized for the Intel Xeon Phi coprocessor.
It pipeline the map and reduce phases to better utilize the hardware resource by producer-consumer model, which has some limitations.
Mao et al.\cite{mao2010metis} consider that the organization of the intermediate values produced by Map invocations and consumed by Reduce invocations is central to achieving good performance on multicore processors. 
They present a optimized MapReduce library (i.e., Metis) using an efficient data structure consisting of a hash table per map thread with a b+tree in each hash entry.
Although, we don't compare \myds with mites, Tilt-MapReduce,
our research shows that if the MapReduce library implemented by Pthreads, there will be problem of scalability.


Serveral paper have looked at scaling systems on mulitcore system.
Boyd-Wickizer et al. aimed at designing a scalable operating system for multicore, named Corey\cite{boyd2008corey}, and presents three new abstractions (address ranges, shares and kernel cores), to scale a MapReduce application running on Corey.
RadixVM\cite{Clements2013RadixVM}, a new virtual memory design
that allows VM-intensive multithreaded applications to scale
with the number of cores.
others ....


