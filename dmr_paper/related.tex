\section{Related Work}
%两个方面:(1)面向多核的mapreduce的研究:Phoenix, mutis, MRPhi (2)linux scalability的问题
\label{sec:rel}
%在写这一部分的时候,需要将全部收集的论文重新看一边,并修改论文中的设计部分,最后,写好abstract, introduction, related work.

Our work is related to the research in programming models for
data-parallel applications, nested data parallelism and multicore re-
lated research. We briefly discuss the most related work in turn.
\subsection{MapReduce programming model}
MapReduce is a popular distributed framework
for massive-scale parallel data analysis.
There are many existing implementation of MapReduce 
which adopt on the basic architecture and 
programming model of originally Google's MapReduce, 
such as Hadoop\cite{}, Dryad\cite{isard2007dryad}.

The Phoenix MapReduce libray
is the most relevant work to \myds. 
Phoenix demostrate that MapReduce is a promising parallel programming
models in muliticore and multiprocess systems.
It creates a thread pool by Pthreads
and can schedule tasks dynamically to support itrative applications.
\myds differs from Phoenix mainly on that 
Phoenix need barrier between iterative MapReduce,
while \myds brokes barrier to speed up computing.

%虽然我们没有与mites, Tilt-MapReduce进行对比,但通过我们的研究,只要这些库使用pthread进行编程,那么就会存在scalability较差的问题。
Tilt-MapReduce\cite{chen2010tiled}
This paper argued that the environmental differences between
clusters and multicore open new design spaces and optimization
opportunities to improve performance of MapReduce on multicore.
Based on the observation, this paper proposed Tiled-MapReduce,
that uses the “tiling strategy” to partition a large MapReduce job
into a number of small sub-jobs and handles the sub-jobs itera-
tively. This paper also explored several optimizations otherwise im-
possible for MapReduce, to improve the memory, cache and CPU
efficiency.

Although, we don't compare \myds with mites, Tilt-MapReduce,
our research shows that if the MapReduce library implemented by Pthreads, there will be problem of scalability.

Metis\cite{mao2010metis},The paper’s main insight is that the organiza-
tion of the intermediate values produced by Map invocations and consumed by Reduce invocations is central
to achieving good performance on multicore processors.
Metis stores these intermediate values using an efficient
data structure consisting of a hash table per Map thread
with a b+tree in each hash entry. As a result, Metis can
achieve better performance than Phoenix on MapReduce
applications that interact with the library frequently (e.g.,
applications with many keys).

MRPhi\cite{} is a multicore mapreduce.

\subsection{Scalability of multicore}
A multicore operating system, named Corey\cite{boyd2008corey}, proposes three
new abstractions (address ranges, shares and kernel cores), to scale
a MapReduce application (i.e., Word Revert Index) running on Corey.
The work in Corey is orthogonal to Ostrich. The abstractions in
Corey, if available in commodity OSes, could further improve the
efficiency of Ostrich due to the reduced time spent in the OS kernel.
	
An Analysis of Linux Scalability to Many Cores \cite{Boyd2010An},
This paper analyzes the scaling behavior of a traditional
operating system (Linux 2.6.35-rc5) on a 48-core com-
puter with a set of applications that are designed for par-
allel execution and use kernel services. We find that we
can remove most kernel bottlenecks that the applications
stress by modifying the applications or kernel slightly.

RadixVM\cite{Clements2013RadixVM}RadixVM, a new virtual memory design
that allows VM-intensive multithreaded applications to scale
with the number of cores.


