\section{Related Work}
%两个方面:(1)面向多核的mapreduce的研究:Phoenix, mutis, MRPhi (2)linux scalability的问题
\label{sec:rel}
%在写这一部分的时候,需要将全部收集的论文重新看一边,并修改论文中的设计部分,最后,写好abstract, introduction, related work.
MapReduce is a popular distributed framework
for massive-scale parallel data analysis.
There are many existing implementation of MapReduce 
which adopt on the basic architecture and 
programming model of originally Google's MapReduce, 
such as Hadoop\cite{}, Dryad\cite{isard2007dryad}.


The Phoenix MapReduce libray\cite{ranger2007phoenix} 
is the most relevant work to \myds. 
Phoenix demostrate that MapReduce is a promising parallel programming
models in muliticore and multiprocess systems.
It creates a thread pool by Pthreads
and can schedule tasks dynamically to support itrative applications.
\myds differs from Phoenix mainly on that 
Phoenix need barrier between iterative MapReduce,
while \myds brokes barrier to speed up computing.

%虽然我们没有与mites, Tilt-MapReduce进行对比,但通过我们的研究,只要这些库使用pthread进行编程,那么就会存在scalability较差的问题。
Although, we don't compare \myds with mites, Tilt-MapReduce,
our research shows that if the MapReduce library implemented by Pthreads, there will be problem of scalability.
