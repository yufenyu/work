\subsection{自我介绍}
我叫俞玉芬,是中国科学技术大学2年级硕士研究生,我的专业是计算机科学与技术,我的课题是面向多核可伸缩的MapReduce的研究。
性格外向,生活态度比较积极,喜欢跑步,骑行,以及户外徒步旅行,还喜欢看一些闲书。工作上,态度认真努力。

\subsection{English introducing}
My name is Yuyufen, coming from AnHui, WuHu. I received the bachelor degree from China
women's university, and now I am pursuing the master degree with the school of computer science and technology in USTC. In the past 2 years, I spend most of my time on study and research.

In my spare time, I like to do some sports, such as, running, riding bicycle. Sports helps me to keep good health and state.

My research interests include: MapReduce model, parallel programming, operator system and computer system, not developing application like Android or IOS. So, I want to do work about operator system and computer system in the future. In my opinion, Intel is an excellent in system field.

I think I am an outgoing person and easy to get well with others, and I have a strong sense of responsibility. When face the challenges and problems, I have a positive attitude and will do my best to solve them.

That’s all. Thank you


\subsection{项目介绍}
我的课题名是“面向多核可扩展性的MapReduce的研究”,我先介绍一下项目的背景。

随着多核机器的广泛普及,如何简单有效地利用多核资源已成为一个十分重要的课题。现有的多核上的并行编程模式,包括共享内存多线程pthread和消息传递MPI,它们需要程序员自己管理线程之间的通信、任务分派和调度,这无形中给程序员增加了很多负担,也让并行编程变的复杂。04年Google提出了MapReduce编程模型,极大的简化了并行编程。受到Google的MapReduce编程思想的启发,耶鲁大学的Range等人将MapReduce编程模型移植到多核环境,他们编写了一套针对多核的MapReduce库,并通过实验证明,应用程序使用MapReduce编程模型在性能上可以与phtread编写的程序相媲美,而且它延续了MapReduce的优点,就是它接口简单,使用方便!

多核环境下的MapReduce库phoenix提出之后,学术界有对phoenix提出了很多的改进和优化。我们通过实验发现,phoenix的可扩展性比较差,具体表现为:应用程序使用phoenix库时,在1-4核环境下,运行时间随着核数的增多而下降;但是8核以上,运行时间不降低,反而上升。特别是在32核环境下,应用程序的运行时间甚至比1核情况的时间都长,非常夸张,也意味着,8核以上的系统,很难在通过phoenix达到理想的性能效果。为了改进这个问题,我的工作是改进phoenix的不可扩展性,重新编写了一套MapReduce模型

涉及的关键技术:MapReduce,线程池的管理,buffer的管理和设计,多线程的编程,进程间的通信,多线程的并发控制问题,

\subsection{MapReduce}
MapReduce是一个编程模型,主要用于处理大的数据集,使用这个模型进行处理时,用户只需要提供两个函数:map和reduce,map函数处理输入的数据,产生中间的key/value键值对;reduce函数将具有相同key的中间结果归并,并产生最终结果。

使用MapReduce编程模型的一个最大优势在于,程序员不需要考虑底层的细节,由运行时系统管理并行,任务的分割,如何在集群环境中进行调度,容错,管理机器间的通信等等。程序员可以不用熟悉并行化,不了解分布式系统的情况下,能够最大化的利用分布式环境的资源。当处理的数据相当大的时候,考虑如何将输入的数据进行分割然后并行的处理,如何充分利用集群环境,如何容错成为关键的问题

\subsection{DMR是如何改进和优化phoenix}
1.phoenix的流程,phoenix的局限和不足,phoenix中影响性能的关键因素

2.新的MapReduce库中的做法,优势


phoenix的处理流程的大致流程也是split, map, reduce,有一个master进程,以及一组worker线程组成的线程池。Phoenix中影响性能的关键因素有两点:1.中间结构是一个全局二维数组,多个map线程和reduce线程都可以访问,为了避免负责的读写控制与同步,以及锁的开销,phoenix使用barrier来简化(具体说一下barrier是如何让编程简单),但barrier严重影响性能。2.Phoenix是使用Pthread编写的多线程,多个线程共享主线程的地址空间,除了栈是私有的。这会带来一个问题,当多个线程在读写一个共享变量时,需要锁进行同步。当线程越多时,花费在锁上的开销会越大, 我们也通过linux perf工具测试了一下,发现CPU的利用率非常低,大部分时候进程都在等待,并没有正在的在cpu上执行。3.cache miss的问题

\subsection{新的MapReduce库针对上述的两个问题进行优化和改进}
1.原有的共享中间结构,变为worker私有的buffer,map将产生的数据存入到自己私有的buffer中,当buffer满的时候,map就将数据发送给reduce,然后继续做map任务,reduce收到数据后,也可以开始reduce的工作。这样做最大的好处是,map和reduce线程间不需要barrier,增加了并行的程度。buffer的实现,还会在cache miss上减少

2.Phoenxi中采用多线程进行编程,多个线程共享master主线程的很多资源,这会造成很多锁的开销,特别是核数越多,线程数越多的情况下,越明显。这是造成不可伸缩性的很重要的因素,为此,在新的MapReduce库中,我们的做法是使用进程。(优势和劣势分别是什么?)进程中的fork是采用COW(解释一下什么是COW机制)的机制创建子进程的,各个进程都拥有自己独立的地址空间,创建时子进程共享父进程的地址空间,但是是只读共享,当它试图修改的时候,就会产生一份自己的拷贝。这样可以避免锁的开销。cpu的利用率会高

3.cache miss会因为使用进程和私有buffer而降低

\subsection{介绍的思路}
简单明了的讲述背景,我的主要工作是优化,重点在于如何去优化phoenix,接下来的一些计划

1.背景: MapReduce, 面向多核的MapReduce库phoenix,我们改进后的MapReduce库

2.phoenix存在的问题:scalability不好,
a.不好的表现是怎样的?
b.目前影响phoenix性能不好的因素
c.scalability不好的原因分析:为了充分利用多核资源phoenix中使用多线程,以充分利用多核资源,进行高度的并发。但多线程存在的问题是什么?


3.从哪几个角度进行改进的:
a.中间数据结构,私有buffer的设计,并尝试不同的buffer的内部实现,破原来的barrier; 
b.针对多线程可扩放性的问题,我们尝试使用进程来完成,但多个进程存在的问题是进程间的通信,编写了一套自己的通道(基于共享内存),底层模型的不一样,我们底层的模型是使用进程来做,不再使用线程(进程和线程的分别优势在于什么)
c.最后的性能表现较好

4.正在考虑和研究的一些问题
a.目前考虑都是从较上层的角度去考虑的,其实内核共享数据结构的竞争,多个cores访问竞争,cache miss的问题,是多核scalability不好的很关键的因素。
b.容错问题

三个大的方面:
1.线程和进程的优缺点的详细对比:

(a)linux中pthread\_creat最终会调用clone(), 进程的创建会调用fork(), 但最终都会调用do\_fork函数进行创建,只是设置的参数不一样。fork()的开销会比较大,比clone()的开销大;(b)由于进程拥有自己的地址空间,而多个线程共享进程的资源,因此线程间的通信会相对较容易,进程则需要多种IPC: pipe, fifo, 互斥锁和条件变量,读写锁,内核常用的spinlock, 信号量,共享内存,消息队列,sorket等方式。(c)调度的时候,进程需要切换的开销比较大,因为进程切换时需要切换pagetable, TLB等,线程切换的开销会相对较小。(d)由于多个线程共享进程的地址空间,当多个线程需要修改同一个位置的位置时,就需要同步,等待的时间会比较长,而进程是采用COW的机制fork子进程的,每个进程都拥有自己独立的地址空间


2.为了改进性能我们做的buffer的改进,并尝试使用buffer的不用实现,array buffer和hash buffer

3.cache miss的降低,cpu的利用率提高

4.多核环境下,多个core对内核中共享数据区的访问也是影响scalability非常重要的因素,目前看来,我们新编写的MapReduce库,16-32情况下是,不增加也不降低,我猜测可能是多个core对内核共享结构访问的造成的,还需要进一步的分析。(内核同步的方式:per-cpu变量,原子操作,memory barrier, spin lock, 读写spin lock,顺序锁,RCU,信号量)

\subsection{DMR的一些缺点和优化点}
虽然DMR的进程地址空间的隔离会让开销变大,但它带来的收益远大于其开销

使用进程模拟线程,本身的开销会比较大,

1.迭代式应用的支持kmeans,线程池。

2.会存在大量key/val的分配,动态内存分配,内存池来实现。



