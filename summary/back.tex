\section{背景}
五个方面:\\
1.确定性并行以及并行编程模型\\
2.分布式环境下的MapReduce,即Google原始的MapReduce,Hadoop,分布式系统\\
3.phoenix,多核环境下的编程模型\\
4.DMR的设计和实现方案\\
5.MapReduce编程模型的不足,衍生为Pregel,spark系统,针对迭代式应用,DMR接下来的改进方式\\

几个问题:\\
1.多核下的MapReduce模型与集群环境下的MapReduce模型的异同对比\\
2.phoenix和DMR的不同支持,DMR的优势和改进在何处\\

\subsection{确定性并行方向}
1.简要介绍课题组目前大的研究方向, 重点介绍我的研究以及我的工作“面向多核可扩展的MapReduce的研究”

2.多核环境下的编程模型, 如何充分利用多核资源,从编程的角度去考虑





