\section{面试总结}

\subsection{网易面试}
关于基础:
剑指offer上面的内容要会手写
进程通信的各种方式
虚拟地址空间的整个转换过程
malloc怎么做的,缺页中断的过程
网络编程试图多了解一些
linux中进程和线程的主要区别,clone()没有本质的区别,那区别到底何在,任务的调度,切换的开销,创建的方式(cow),地址空间等
自旋锁和互斥锁有什么区别,进程或者线程间同步的方式有哪些?读写者问题如何保证公平性,如果写者进程一直得不到锁怎么办,会出现什么状况?饥俄最后饿死
pthread\_mutex是怎么实现的?

虚函数是怎么实现的

关于项目:
1.可以总结的更加清楚,同时也要更加有深度,不能让人觉得你的东西很简单,没什么深度,有深度,而且非常清晰
2.扩展问题和技术:memory barrier是什么,CPU的亲和性, 有没有考虑cache miss的问题
3.为什么一定要用进程而不用线程?你说可以避免竞争,但线程需要竞争的,进程也同样需要竞争,你不能通过使用线程来避免进程。关于SPMC模型
4.分布式环境下的MapReduce,以及分布式环境
5.人家会问,容错问题,你是怎么考虑的?直接说,这部分我目前没有考虑,肯定不好,你打算怎么做,思路,分布式环境下的容错是怎样的?
6.是否有写技术博客

HR面:
1.本科不是很好,考研
2.大学毕业后的两年做什么了,是否觉得这两年是一种浪费(不会,为什么,1.学到了很多,2.更加珍惜读研的时间和现在的机会,是自己付出很多汗水换来的,经历让人懂得珍惜)
3.你跟别人相比最大的优势在什么地方?(悟性好,团队合作)
4.你跟别人对比劣势在什么地方?
