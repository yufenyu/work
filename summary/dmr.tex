\section{DMR}
\subsection{DMR的实现流程}

\subsection{DMR的优势和局限性}
DMR相比Phoenix的优势在哪?\\
1.DMR打破原有phoenix的barry,而barry在phoenix中占有的开销非常大\\
2.DMR中使用进程模拟线程,worker之间是地址隔离的,当存在共享数据时,便降低开销\\
因此DMR对访存密集型的应用程序有很大的优势,实验的结果是,随着核数的怎多,在64核环境下,是phoenix的10倍
对于计算密集型的应用,它的优势不是很大

DMR的局限性:\\
1.目前的DMR,无法很好的支持迭代式应用,比如典型的kmeans聚类计算,其根本的原因是,phoenix中采用的是线程池,而DMR中,则是每次MapReduce计算时都需要创建和销毁环境,通过具体的实验发现,kmeans在32核环境下,创建和销毁环境的开销占用了50\%以上的开销。

2.进程的间的地址隔离带来的挑战和开销
