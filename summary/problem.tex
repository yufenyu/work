\section{面试问题汇总}

\subsection{算法和数据结构}
1.二叉树,B+树,AVL树,红黑树,哈夫曼树,堆是什么

2.常用hash算法有哪些

3.查找算法,各种算法的时间和空间复杂度

4.排序算法,各种算法的时间和空间复杂度

5.如何判断单链表是否有环?

6.二叉树的遍历,递归和非递归的写法,层次遍历的算法

7.图的BFS与DFS算法,最小生成树算法,最短路径Dijkstra算法

8.KMP算法

9.排列组合问题,如何求解

10.大数据处理

\subsection{操作系统}
1.锁的优化策略\\
① 读写分离\\
② 分段加锁\\
③ 减少锁持有的时间\\
④ 多个线程尽量以相同的顺序去获取资源\\

2.操作系统在生命情况下会死锁,死锁发生的比较条件?死锁预防?死锁避免?

3.进程间通信(IPC)有哪几种方式?
管道(PIPO),命名管道(FIFO), 信号, 消息队列,共享内存,信号量,套接口(socket)
(a)消息传递:管道,FIFO,消息队列
(b)同步:互斥锁,条件变量,读写锁,信号量
(c)共享内存区:匿名共享内存区,有名共享内存区
(d)过程调用:RPC

4.线程同步,线程阻塞

5.操作系统如何进行分页调度

6. 什么是虚拟内存

7. 虚拟地址、逻辑地址、线性地址、物理地址的区别。

8.进程和线程的区别
\subsection{C或C++}
1.C++里面的几个cast

2.c里面的valatile关键字
\subsection{网络}
1.http和https的主要区别,HTTP的报文结构,HTTP的状态码含义,HTTP request的集中类型,Http 1.1和Http1.0的区别?

2.cookie和session的区别以其原理分别是什么

3.TCP如何保证安全性,TCP如何保证可靠传输?三次握手过程?四次挥手过程?过程中各个状态名称和含义,TIMEWAIT为什么需要三次握手,两次为什么不可以?TCP报文结构?\\
TCP可靠性:TCP/IP p170; 三次握手,四次挥手:计算机网络 p216;
为什么要三次握手,两次不行:考虑"已失效的链接请求报文段",如果一个已失效的请求发送的服务器,然后建立链接,而事实并不需要,这样便浪费资源。\\
为什么A在TIMEWAIT状态必须等待2MSL的时间:1.保证A发送的最后一个ACK能够到达B 2.防止“已失效的链接请求报文段”出现在本链接中

4.Get和post的区别

5.TCP和UDP区别?UDP的应用场景,为什么\\
(a)TCP是可靠的,传输速度满,UDP是不可靠的,传输速度快,尽最大努力交付
(b)TCP是面向字节流的,UDP是面向报文的
(c)TCP的报文头部有20字节,UDP头部8字节,开销小
(d)TCP:(FTP:21, Telnet:23, SMTP:25, http:80);UDP:(DNS:53, TFTP:69, SNMP:161, RIP:520,NFS(网络文件系统))

总结:TCP:面向连接、传输可靠(保证数据正确性,保证数据顺序)、用于传输大量数据(流模式)、速度慢,建立连接需要开销较多(时间,系统资源)。UDP:面向非连接、传输不可靠、用于传输少量数据(数据包模式)、速度快。

TCP的使用场景:当对网络通讯质量有高要求的时候,比如:整个数据要准确无误的传送给对方,这往往用于一些要求可靠的应用,比如HTTP,HTTPS,FTP等传输,POP,SMTP等邮件传输的协议

UDP的使用场景:当网络通讯质量要求不高的时候,要求网络通讯速度能尽量的快,这时就可以使用UDP,比如QQ视频,QQ语音

6.滑动窗口算法?

7.TCP的拥塞处理:慢启动,拥塞避免,拥塞发生,快恢复

8.从输入网址到获取页面的过程:查询DNS,获取域名对应的IP地址(DNS的查询过程),浏览器获得域名对应的IP地址,发起HTTP三次握手,TCP/IP链接建立起来后,浏览器就可以向服务器发送HTTP请求,服务器接受请求,将结果返回给浏览器,DNS, HTTP, TCP, OSPF, IP, ARP

9.ping的整个过程,ICMP报文是什么

10.路由器和交换机的区别


