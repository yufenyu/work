
本周主要工作:\\
  1.测试参数DEFAULT\_VALS\_ARR\_LEN(key对应的value数据的长度),
  MAX\_LEN(buffer中发送数据的阈值),不同的值对性能的影响情况\\
  2.完成map阶段的buffer的数组实现\\
  3.更新dmr-1.0的设计文档git中papers/determ/spmc/mapreduce/design\\

{\color{red}下周计划:}目前数据的实现只是正对word\_count改进,其他应用程序无法支持,
为此需要改进buffer的数组实现
\begin{enumerate}
  \item 修改insert\_keyva\_map()(目前并没有调用combiner,仅仅是value=value+1)
  \item 支持有key或者value的copy情况
  \item 支持不开启combiner的情况
  \item 调整代码以文件结构,方便编译不同版本的DMR
\end{enumerate}

\subsection{深度combiner}
1.常量DEFAULT\_VALS\_ARR\_LEN,每个key默认的value数组的长度。在开启combiner情况下,
value数组的长度为DEFAULT\_VALS\_ARR\_LEN时,便进行combiner;
否则,realloc一块更大的空间存放value

2.实验结果显示:DEFAULT\_VALS\_ARR\_LEN=2比DEFAULT\_VALS\_ARR\_LEN=5的性能要好
通过观察详细的数据,发现值为2的情况下,
group\_keyval,map\_send, reduce\_recv, insert\_keyval, mem\_map的次数明显降低,
从而提高性能。

3.分析原因\\
如图\ref{kv_len}所示,不同的DEFAULT\_VALS\_ARR\_LEN,将直接影响group\_keyval,
在2的情况下,由于更加频繁的调用combiner,每个key最多有两个value,
因此group时,需要分配更少的空间,遍历整个buffer花费的时间也会更少,
最后通过通道发送数据时,发送的次数少
\begin{figure}[!h]
    \centering
    \includegraphics[height=4cm,width= 12cm]{img/dmr-kv_len.png}
    \caption{不同DEFAULT\_VALS\_ARR\_LEN的影响}
\label{kv_len}
\end{figure}


\subsection{发送的粒度}
常量MAX\_LEN表示当buffer中的key/val的个数超过该阈值时,
便将buffer中的数据发送出去,channel底层的实现时,如果采用lazy方式,
那么凑满一页便发送出去。

如果MAX\_LEN设置不合理,可能导致一条消息被分成两次发送,由于采用lazy策略,
第二条消息需要满一页再发送,这样接受端被阻塞

测试的结果显示MAX\_LEN为160时具有较好的性能

\subsection{实验结果分析}
经过之前的对DMR的优化,在32核centos系统重新进行测试,发现
\begin{itemize}
  \item 1和2线程情况下,开销分别比phoenix高出2.8倍和1.4倍
  \item 4到32线程情况下,性能都非常的好,且随着线程数的增多,性能越好
\end{itemize}

为了重点分析1和2线程下性能差的具体原因,收集了word\_count运行阶段的各部分的详细时间开销,
同时收集phoenix的各部分的时间开销,发现group\_keyval占据了很大的时间,特别是在1线程下,
已经超过总运行时间的50\%

为了从根本上减少group带来的额外开销,考虑是否可以不使用group,
那便意味这不再使用hash表实现buffer,而是改用数组

\subsection{buffer的数据实现}
默认情况下的buffer是hash表实现的,由于我们采用了分层的设计方案,
为了修改buffer的具体实现,只需要修改mapbuf.c即可,保持mapbuf.h的接口不变

1.array实现,每个buffer的结构如图\ref{buf_array}
\begin{figure}[!h]
    \centering
    \includegraphics[height=2cm,width= 8cm]{img/buf_array.png}
    \caption{buffer implemented by array}
\label{buf_array}
\end{figure}
初始化buffer时,分配MAX\_LEN长度的数据,之后重复利用,而无需再进行多次动态分配。
mapper产生一个key/val,首先计算对应的hash值,根据hash值得出key的插入位置,
将key/val对插入

2.发送
由于多个key已经组织在一个array中,当buffer满时(即key/val数等于MAX\_LEN时),
便发送数据,无需group,发送结束后,只需将array的长度设置为0即可。

3.使用array实现的优势和局限性
\begin{itemize}
  \item 优势:相比hash表的实现,省去发送前的group操作,提高性能
  \item 局限性:会增加查找key,memmove以及发送chan\_send的开销,且目前不支持key,value的copy,以及value不是整型。
\end{itemize}

4.实验结果,32centos上的实验结果显示,针对word\_count,
buffer采用数据实现的性能非常好,如图所示\ref{wc}
\begin{figure}[!h]
    \centering
    \includegraphics[height=5cm,width= 9cm]{img/wc.png}
    \caption{word\_count results}
\label{wc}
\end{figure}




