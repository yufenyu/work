\section{描述}
我的课题名是“面向多核可扩展性的MapReduce的研究”,最终的目标是编写一个可扩展性的MapReduce库,成为多核环境下的编程模型。

一方面,MapReduce是Google公司提出的一种编程模型,利用集群资源处理大的数据集。这个模型的最大优势是简单,程序员不需要考虑底层的细节,比如任务的分割,多个任务的并写执行,如何在集群环境中进行调度,容错,管理机器间的通信等等,这些细节都是由MapReduce运行时系统管理。程序员可以不用熟悉并行化,不了解分布式系统的情况下,也能够使用MapReduce编程模型最大化的利用分布式环境的资源。

程序员只需要提高两个函数:map和reduce;map函数对输入数据进行处理,得到key/value的键值对,这些键值对,成为map产生的中间结果;reduce函数的作用是,将具有相同key的中间结果归并成,得到最终结果。

另一方面,随着多核机器的广泛普及,如何充分发挥多核系统的资源,成为一个很重要的课题,其中,从编程模型的角度去思考,是一个很重要的方式,因此Google提出的MapReduce编程模型后,耶鲁大学的Range等人将MapReduce编程模型移植到多核环境,他们编写了一套针对多核的MapReduce库,并通过实验证明,MapReduce编程模型在性能上可以与phtread编写的程序想媲美,而它优于pthread的是,它简单!

我的工作是,基于我们课题组已有的一个虚拟内存模型,重写编写了一套针对多核的MapReduce库,我们成为DMR,通过实验结果


