\section{描述}
我的课题名是“面向多核可扩展性的MapReduce的研究”,课题的目标是编写一个可扩展性的MapReduce库,提供简洁高效的编程。先从两个方面介绍该课题。第一是MapReduce编程模型,第二是多核环境下的MapReduce

一方面,MapReduce是Google公司提出的一种编程模型,利用集群资源处理大的数据集。这个模型的最大优势是简单,也就是说,程序员不需要考虑底层的细节,比如任务的分割,多个任务的并写执行,如何在集群环境中进行调度,如何容错,如何管理机器间的通信等等问题,这些细节复杂的问题都是由MapReduce运行时系统进行管理。程序员可以不用熟悉并行化,不了解分布式系统的情况下,也能够使用MapReduce编程模型最大化的利用分布式环境的资源,进行大数据的计算。

程序员在使用MapReduce编程模型时,只需要提高两个函数:map和reduce;map函数对输入数据进行处理,得到key/value的键值对;reduce函数是对这些键值对进行处理,将具有相同key的多个value进行归并,得到最终结果。


另一方面,随着多核机器的广泛普及,如何充分发挥多核系统的资源,成为一个很重要的课题,其中,从编程模型的角度去思考,是一个很重要的方式,因此Google提出的MapReduce编程模型后,耶鲁大学的Range等人将MapReduce编程模型移植到多核环境,他们编写了一套针对多核的MapReduce库,并通过实验证明,MapReduce编程模型在性能上可以与phtread编写的程序想媲美,而且它延续了MapReduce的优点,就是它简单!

我的工作是,基于我们课题组已有的一个虚拟内存模型,重写编写了一套针对多核环境的MapReduce库,我们称为DMR。

\subsection{多核环境下的MapReduce库与集群环境下的MapReduce库的异同}
多核环境下的MapReduce与集群环境下的MapReduce对比。

\subsubsection{不同之处}
1.不同的runtime\\
不论是多核环境还是集群环境,MapReduce编程模型的处理流程大致是这样的:
\begin{itemize}
  \item 首先对输入的数据进行分割成多个子任务,放入任务队列,该阶段成为split过程
  \item 然后进入map阶段,map阶段会有多个worker,每个worker会从任务队列中取任务,并调用用户提供的map函数进行map计算,产生key/val的键值对,map阶段产生的结果,称为中间结果
  \item 等到所有map worker都完成任务后,便开始reduce阶段,reduce阶段对map产生的中间结果进行归并,每个reduce会得到最终的结果的片段。
  \item 最后merge阶段,merge会将多个reduce产生的局部结果进行归并,成为最终的结果,返回给应用程序。
\end{itemize}
刚刚说的,是MapReduce模型中的主要过程,其中还有一些可选的结果,比如combiner阶段

虽然多核环境和集群环境下的MapReduce,都遵刚刚所讲述的过程,但环境不同,具体的运行时也是不一样的,主要表现在几个方面:
\\
第一:worker的不同形式
\begin{itemize}
  \item 在多核环境下,master主线程管理一个子线程组成的线程池,每个线程在运行的时候,与cpu上的一个core进行绑定,然后执行任务。
  \item 在集群环境下,master也会fork很多子线程,但每个slaver都是一个单独的机器上运行。
\end{itemize}

第二:中间结果的存放
\begin{itemize}
  \item 在多核环境下,map产生的中间键值对会被直接存放到内存的共享区,接下来reduce从中取,进行reduce操作,最终的结果存入到内存。
  \item 在集群环境下,map产生的中间结果会存放到本地磁盘中,然后通知master,当master安排reduce任务时,reduce会根据master发来的信息,从对应的map的disk中取数据。reduce最终的结果会存入到分布式文件系统GFS.
\end{itemize}


第二:影响性能的主要因素不同。
\begin{itemize}
  \item 多核环境下的MapReduce:性能的主要限制在于中间数据结构的存储,即用于存储key/val的数据结构。因为多个map线程和多个reduce线程有可能同时访问该共享结果,这会涉及并发控制的问题,通常表现为在锁的等待上的开销比较大。phoenix对该部分进行改进,每个线程操作的区域不同,可以避免在锁的等待上话费的开销。
  \item 集群环境下的MapReduce:最关注的是网络带宽资源,因为worker和master之间的通信是通过网络,降低网络带宽是集群环境下MapReduce考虑的主要问题。为此MapReduce做出的努力有如下几方面:
  
\begin{itemize}
  \item Map input is read from local disks, not over network,尽量在存放input replication上启动map worker,这样便可以直接从local disk读取,而不是通过网络(moving computation to data)
  \item Map阶段的中间结果存放于local disk而不是GFS中,为什么?
  \item Intermediate data partitioned into files holding many keys,Big network transfers are more efficient
  \item map阶段的combiner操作,所谓combiner就是局部的reduce操作,将一个key对应的多个value进行局部的归并,这样可以有小减少网路通信量,以及本地磁盘空间的使用,以word\_count为例。
\end{itemize}
  
\end{itemize}


\subsubsection{相同之处}
1.基本的思想和处理流程是一样的


2.负载均衡的问题

3.优势相同:简单的编程模型
MapReduce编程模型隐藏了运行时的各种细节,程序员在使用该编程模型的时候,无需考虑并行,负责均衡,资源是否充分被利用,容错等问题,这些都由MapReduce库来管理。用户只需要提供map和reduce两个函数逻辑即可。

\subsection{DMR的优势与局限}

\subsection{DMR与phoenix有何不同}

\subsection{MapReduce编程模型的不足之处}


