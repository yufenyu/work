\section{MapReduce}
现实中很多应用可以归结为这样一个模型,
MapReduce是一个编程模型,主要用于处理大的数据集,使用这个模型进行处理时,用户只需要提供两个函数:map和reduce,map函数处理输入的数据,产生中间的key/value键值对;reduce函数将具有相同key的中间结果归并,并产生最终结果。

使用MapReduce编程模型的一个最大优势在于,程序员不需要考虑底层的细节,由运行时系统管理并行,任务的分割,如何在集群环境中进行调度,容错,管理机器间的通信等等。程序员可以不用熟悉并行化,不了解分布式系统的情况下,能够最大化的利用分布式环境的资源

当处理的数据相当大的时候,考虑如何将输入的数据进行分割然后并行的处理,如何充分利用集群环境,如何容错成为关键的问题

MapReduce的最终目标是
hides the messy details of parallelization, fault-tolerance, data distribution and load balancing in a library.同时具有相当好的性能

MapReduce的实现方式有很多,主要看具体的机器和运行环境,Google的实现提供的是一种实现方式

\subsection{实现}
\subsubsection{MapReduce的执行流程}
1.user program首先会将输入数据分割成M份,然后在集群的机器中fork很多子程序。
2.多个程序中,有一个是master,其他都是worker,master会挑选一个空闲的worker,让它执行map任务或者reduce任务
3.执行map任务的worker,首先读取输入的内容,调用用户提供的map函数,产生中间键值对,将这些键值对存入到内存中。
4.map会周期性将内存中的内容写入到硬盘,然后将位置信息发送给master,然后master发送这些信息给reducer
5.reduce接受到master发来的通知后,它会使用RPC(remote procedure calls)从map的disk中读取数据,然后reduce对读到的数据进行排序,这样相同key会在一起
6.reduce调用用户提供的reduce函数,数据进行归并
7.当所有的map和reduce任务完成以后,master唤醒user program,mapreduce库结束,返回user code

\subsubsection{Fault Tolerance}
1.worker failure
如果mapper出错,那么需要重新执行map task,因为不能从fail的机器中取出disk中的中间结果
如果reducer出错,那么不需要重新执行整个reduce task,因为reducer的结果存放于global file system
如果一个map task一开始执行于worker A,之后执行于worker B,那么所有的reducer就需要重新执行,因为reducer需要从B读取中家结果

2.master failure
目前的MapReduce中,因为只有一个master,还不能容错

How does MR cope with worker crashes?
  * Map worker crashes:
    master re-runs, spreads tasks over other GFS replicas of input.
      even if worker had finished, since still need intermediate data on disk.
    some Reduce workers may already have read failed worker's intermediate data.
      here we depend on functional and deterministic Map()!
    how does the master know the worker crashed? (pings)
    master need not re-run Map if Reduces have fetched all intermediate data
      though then a Reduce crash would have to wait for Maps to re-run
  * Reduce worker crashes before producing output.
    master re-starts its tasks on another worker.
  * Reduce worker crashes in the middle of writing its output.
    GFS has atomic rename that prevents output from being visible until complete.
    so it's safe for the master to re-run the Reduce tasks somewhere else.

Other failures/problems:
  * What if the master accidentally starts *two* Map() workers on same input?
    it will tell Reduce workers about only one of them.
  * What if two Reduce() workers for the same partition of intermediate data?
    they will both try to write the same output file on GFS!
    atomic GFS rename will cause the second to finish to win.
  * What if a single worker is very slow -- a "straggler"?
    perhaps due to flakey hardware.
    master starts a second copy of last few tasks.
  * What if a worker computes incorrect output, due to broken h/w or s/w?
    too bad! MR assumes "fail-stop" CPUs and software.
  * What if the master crashes?

\subsubsection{Locality}
网络带宽是非常稀缺的资源,因此应当尽量减少网络通信量。Google的MapReduce使用了使用GFS作为文件的管理工具,输入文件存就放在集群中,MapReduce master会根据GFS提供的输入文件的copy的位置,并试图在存有输入文件的机器上启动任务,这样便可以避免网络的传输。
schedule a map task on a machine that contains a replica of the corresponding input data.
如果没有,那么会从最近的备份的机器中找。这样便具有好的局部性(Locality)

\subsubsection{Task Granularity}
map阶段的M pieces以及reduce阶段的R pieces,远远大于woker机器的数量,这样做的好处是,可以提高dynamic load balancing

\subsubsection{Backup Tasks}
map和reduce之间有一个barry,如果有一个map或者reduce特别慢,那么会增加这个MapReduce计算的total time
只有当map任务全部完成后,才会启动reduce任务,否这reduce无法获取全局的值
reduce的最终结果会写入到GFS

\subsection{Refinements}
几处可选的优化操作
\subsubsection{partition function}
reduce有R个,map产生的中间结果根据hash(key)\%R被发送给reduce,这样对于负载平衡有很大的好处

\subsubsection{combiner function}
将map产生的中间结果,进行局部的reduce操作,这样可以有小减少网路通信量,以及本地磁盘空间的使用
以word count为例,同一个key有1000个value,如果每个都是<key,1>那么需要存放1000次,需要通过网络发送1000次,如果进行局部的reduce那么就成为<key,1000>只要存放一个,然后发送一次

\subsection{总结}
MapReduce Model易于编程:it hides many painful details:
concurrency -- same result as sequential execution
starting s/w on servers
data movement
failures

This model scales well
map不需要等待其他的map也不需要共享数据,这样便可以并发的执行
reduce也一样

what will be the limiting factor in performance?
可以优化的地方在哪里?CPU? memory? disk? network?
They were limited by "network cross-section bandwidth",事实上,分布式环境下,最敏感的资源便是网络带宽,传输的量,已经网络延迟
但网络是外界的资源,很难去优化,我们能够做的是,尽量减少网络通信量
MapReduce做出的努力
\begin{itemize}
  \item Map input is read from local disks, not over netwokr,尽量在存放input replication上启动map worker,这样便可以直接从local disk读取,而不是通过网络
  \item Map阶段的中间结果存放于local disk而不是GFS中,为什么?
  \item Intermediate data partitioned into files holding many keys,Big network transfers are more efficient
\end{itemize}

负载不均衡
How do they get good load balance?
理想的状态是每个server都能有相同的处理能力,然后或许相同的工作,最大的并发执行.
But time to process a split or partition isn't uniform.
Different sizes and contents, and different server hardware

解决的方案:many more splits than workers
Master hands out new splits to workers who finish previous tasks.
so no split is so big it dominates completion time.
这样的话,faster servers do more work than slower ones, finish about the same time

For what applications *doesn't* MapReduce work well?
  Not everything fits the map/shuffle/reduce pattern.
  Small data, since overheads are high. E.g. not web site back-end.
  Small updates to big data, e.g. add a few documents to a big index
  Unpredictable reads (neither Map nor Reduce can choose input)
  Multiple shuffles, e.g. page-rank (can use multiple MR but not very efficient)
  More flexible systems allow these, but more complex model.

因此MapReduce最大的优势是简单,但这种简单也限制了它对更多复杂应用程序的支持





