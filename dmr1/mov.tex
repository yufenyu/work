\section{dmr-1.0设计概述和研究动机}
dmr-1.0是对原来的最初版本的DMR进行改写,
论文的关键点:DMR较好的scalability。
整体的思路:
\begin{itemize}
  \item 观察和分析Phoenix scalability不好的原因
  \item 基于新的Producer-Consumer模型搭建DMR.
  \item 优化性能:buffer的重新设计与实现
\end{itemize}


%\section{研究动机}
随着多核机器的广泛普及,如何简单有效地利用多核资源已成为一个十分重要的课题。现有的多核上的并行编程模式,包括共享内存多线程pthread和消息传递MPI,它们需要程序员自己管理线程之间的通信、任务分派和调度,这无形中给程序员增加了很多负担,也让并行编程变的复杂。

2004年Google提出了MapReduce编程模型,极大的简化了并行编程。受到Google的MapReduce编程思想的启发,耶鲁大学的Range等人将MapReduce编程模型移植到多核环境,他们编写了一套针对多核的MapReduce库,并通过实验证明,应用程序使用MapReduce编程模型在性能上可以与phtread编写的程序相媲美,而且它延续了MapReduce的优点,就是它的编程接口简单。用户可以不用熟悉多核环境,也不需要了解任务的分割和并行执行等问题,只需要提供简单的map函数和reduce函数,就能够最大化的利用多核资源。

多核环境下的MapReduce库phoenix提出之后,学术界对phoenix提出了很多的改进和优化,例如:Phoenix2, Metis,phoenix++, MALK。我们通过实验发现,phoenix的可扩展性比较差,具体表现为:应用程序使用phoenix库时,在1-8核环境下,运行时间随着核数的增多而下降;但是8核以上,运行时间不降低,反而上升,特别是在32核环境下,应用程序的运行时间甚至比1核情况的时间都长。这种不稳定性意味着,8核以上的系统,很难通过使用phoenix达到理想的性能效果。为了改进这个问题,我们重新编写了一套MapReduce库,它给用户提供的接口与phoenix一致,因此phoenix的应用程序几乎不需要修改就可以使用我们的MapReduce库。
